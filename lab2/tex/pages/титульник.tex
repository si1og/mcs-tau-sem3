{\thispagestyle{empty}
\begin{center}
    \hfill \break
    \hfill \break
    Министерство образования и науки Российской Федерации\\[3pt]
    Санкт-Петербургский политехнический университет Петра Великого\\[10pt]
    Институт компьютерных наук и кибербезопасности\\[3pt]
    Высшая школа технологий искусственного интелекта\\[3pt]
    Направление: 02.03.01 «Математика и компьютерные науки»\\
    [110pt]

    \large Расчётное задание №2 \\[5pt]
    \large <<Устойчивость линейной системы автоматического управления>> \\[5pt]
    \large по дисциплине: <<Теория автоматического управления>>\\[15pt]
    {\Huge i=17}
\end{center}

\vspace{90 pt}

\begin{tabular*}{460pt}{@{\extracolsep{\fill}} l r l}
    Выполнил студент\tabularnewline группы №3 & \hspace{50pt} \line(1,0){100} \hspace{-75pt} & Семенов И. А.\\
     & \\
     Проверил \tabularnewline преподователь & \hspace{50pt} \line(1,0){100} \hspace{-75pt} & Суханов А. А. \\
     & \\
     & \\
     & \\
     & \multicolumn{2}{r}{\guillemotleft \line(1,0){30} \guillemotright \line(1,0){100} \, 2025г.}
     
\end{tabular*} \\

\vfill


\centering Санкт-Петербург

\centering осень, 2025

\newpage
}