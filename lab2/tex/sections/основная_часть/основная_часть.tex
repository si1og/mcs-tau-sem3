\section*{Решение}

\subsection*{1. Характеристическое уравнение замкнутой системы}

\begin{itemize}
  \item По условию дана передаточная функция разомкнутой системы.
  \item Вычислим передаточную функцию замкнутой системы.
\end{itemize}

Рассмотрим систему, замкнутую единичной обратной связью:

\begin{center}
\begin{tikzpicture}[>=stealth, node distance=2cm]
    \node (input) at (0,0) {};
    \node[draw, circle, minimum size=0.6cm] (sum) at (1.5,0) {};
    \node[draw, rectangle, minimum width=2cm, minimum height=1cm] (system) at (4,0) {$H_{\text{раз}}$};
    \node (output) at (6.5,0) {};
    
    \draw (1.29,0.21) -- (1.71,-0.21);
    \draw (1.29,-0.21) -- (1.71,0.21);
    
    \draw[->] (0,0) -- node[above] {$u$} (sum);
    \draw[->] (sum) -- (system);
    \draw[->] (system) -- node[above] {$y$} (6.5,0);
    \draw[->] (6,0) -- (6,-1.2) -- (1.5,-1.2) -- (sum);
\end{tikzpicture}
\end{center}

Передаточная функция замкнутой системы:
\[
H_{\text{зам}}(p) = \frac{H_{\text{раз}}(p)}{1 + H_{\text{раз}}(p)}.
\]

Подставим выражение для $H_{\text{раз}}(p)$:
\[
H_{\text{зам}}(p) = \frac{\dfrac{K(T_1 p + 1)}{p(p^2 + \omega_0^2)(T_2 p + 1)(T_3 p + 1)}}{1 + \dfrac{K(T_1 p + 1)}{p(p^2 + \omega_0^2)(T_2 p + 1)(T_3 p + 1)}}
\]
\[
= \frac{K(T_1 p + 1)}{p(p^2 + \omega_0^2)(T_2 p + 1)(T_3 p + 1) + K(T_1 p + 1)}.
\]

Характеристический полином замкнутой системы $\alpha(p)$ — это знаменатель передаточной функции:
\[
\alpha(p) = p(p^2 + \omega_0^2)(T_2 p + 1)(T_3 p + 1) + K(T_1 p + 1).
\]

Подставим значения параметров для $i = 17$:
\[
K = 4.25, \quad \omega_0 = 17, \quad T_1 = 0.01, \quad T_2 = 0.34, \quad T_3 = 0.17.
\]

Тогда:
\[
\alpha(p) = \underbrace{p(p^2 + 289)(0.34p + 1)(0.17p + 1)}_{(1)} + \underbrace{4.25(0.01p + 1)}_{(2)}.
\]

Раскроем скобки поэтапно.

Вычислим (1):
\begin{align*}
(1) &= p(p^2 + 289)(0.34p + 1)(0.17p + 1) \\
&= p(p^2 + 289)(0.0578p^2 + 0.51p + 1) \\
&= p(0.0578p^4 + 0.51p^3 + p^2 + 16.7042p^2 + 147.39p + 289) \\
&= p(0.0578p^4 + 0.51p^3 + 17.7042p^2 + 147.39p + 289) \\
&= 0.0578p^5 + 0.51p^4 + 17.7042p^3 + 147.39p^2 + 289p.
\end{align*}

Вычислим (2):
\begin{align*}
(2) &= 4.25(0.01p + 1) \\
&= 0.0425p + 4.25.
\end{align*}

Сложим (1) и (2):
\begin{align*}
\alpha(p) &= (1) + (2) \\
&= 0.0578p^5 + 0.51p^4 + 17.7042p^3 + 147.39p^2 + 289p + 0.0425p + 4.25 \\
&= 0.0578p^5 + 0.51p^4 + 17.7042p^3 + 147.39p^2 + 289.0425p + 4.25.
\end{align*}

Таким образом, получили характеристический полином пятого порядка:
\[
\alpha(p) = 0.0578 p^5 + 0.51 p^4 + 17.7042 p^3 + 147.39 p^2 + 289.0425 p + 4.25,
\]

где коэффициенты:
\begin{gather*}
  a_0 = 0.0578, \quad a_1 = 0.51, \quad a_2 = 17.7042, \quad a_3 = 147.39, \\
  a_4 = 289.0425, \quad a_5 = 4.25.
\end{gather*}

\subsection*{2. Критерий устойчивости Гурвица}

Для анализа устойчивости системы применим алгебраический критерий Гурвица.

Согласно теореме Гурвица, для того чтобы у вещественного многочлена $\alpha(p) = a_0 p^n + a_1 p^{n-1} + \ldots + a_n$ все корни имели отрицательные вещественные части, необходимо и достаточно, чтобы
\[
\begin{cases}
\mathrm{sign}\, \Delta_k = \mathrm{sign}\, a_0, & k = 2m - 1 \\
\Delta_k > 0, & k = 2m
\end{cases} \quad (m \in \mathbb{N}),
\]
где $\Delta_k$ — ведущие главные миноры $\Gamma$ порядка $k$ (определители Гурвица).

Поскольку $a_0 = 0.0578 > 0$, условие сводится к $\Delta_k > 0$ для всех $k = 1, 2, \ldots, 5$.

Составим матрицу Гурвица для полинома 5-го порядка:
\[
\Gamma = \begin{pmatrix}
a_1 & a_3 & a_5 & 0 & 0 \\
a_0 & a_2 & a_4 & 0 & 0 \\
0 & a_1 & a_3 & a_5 & 0 \\
0 & a_0 & a_2 & a_4 & 0 \\
0 & 0 & a_1 & a_3 & a_5
\end{pmatrix}.
\]

Подставим значения коэффициентов:
\begin{gather*}
  a_0 = 0.0578, \quad a_1 = 0.51, \quad a_2 = 17.7042, \quad a_3 = 147.39, \\
  a_4 = 289.0425, \quad a_5 = 4.25.
\end{gather*}


\[
\Gamma = \begin{pmatrix}
0.51 & 147.39 & 4.25 & 0 & 0 \\
0.0578 & 17.7042 & 289.0425 & 0 & 0 \\
0 & 0.51 & 147.39 & 4.25 & 0 \\
0 & 0.0578 & 17.7042 & 289.0425 & 0 \\
0 & 0 & 0.51 & 147.39 & 4.25
\end{pmatrix}.
\]

Вычислим определители Гурвица.

\textbf{Определитель $\Delta_1$:}
\[
\Delta_1 = a_1 = 0.51 > 0.
\]

\textbf{Определитель $\Delta_2$:}
\begin{align*}
  \Delta_2 = \begin{vmatrix}
  a_1 & a_3 \\
  a_0 & a_2
  \end{vmatrix} = a_1 a_2 - a_0 a_3 &= 0.51 \cdot 17.7042 - 0.0578 \cdot 147.39 \\
                                    &= 9.029142 - 8.519142 = 0.51 > 0.
\end{align*}

\textbf{Определитель $\Delta_3$:}
\begin{align*}
\Delta_3 = \begin{vmatrix}
a_1 & a_3 & a_5 \\
a_0 & a_2 & a_4 \\
0 & a_1 & a_3
\end{vmatrix} &= a_3 \Delta_2 - a_1 (a_1 a_4 - a_0 a_5) \\
&= 147.39 \cdot 0.51 - 0.51 \cdot (0.51 \cdot 289.0425 - 0.0578 \cdot 4.25) \\
&= 75.1689 - 0.51 \cdot (147.411675 - 0.24565) \\
&= 75.1689 - 0.51 \cdot 147.166025 \\
&= 75.1689 - 75.05467275 = 0.11422725 > 0.
\end{align*}

\textbf{Определитель $\Delta_4$:}
\begin{align*}
\Delta_4 = \begin{vmatrix}
a_1 & a_3 & a_5 & 0 \\
a_0 & a_2 & a_4 & 0 \\
0 & a_1 & a_3 & a_5 \\
0 & a_0 & a_2 & a_4
\end{vmatrix} &= a_4 \Delta_3 - a_5 \begin{vmatrix}
a_1 & a_3 & a_5 \\
a_0 & a_2 & a_4 \\
0 & a_0 & a_2
\end{vmatrix}.
\end{align*}

Вычислим вспомогательный определитель:
\begin{align*}
\begin{vmatrix}
a_1 & a_3 & a_5 \\
a_0 & a_2 & a_4 \\
0 & a_0 & a_2
\end{vmatrix} &= a_1(a_2^2 - a_0 a_4) - a_3(a_0 a_2) + a_5(a_0^2)\\
&= 0.51 \cdot (17.7042^2 - 0.0578 \cdot 289.0425) - \\
&\phantom{= (} - 147.39 \cdot (0.0578 \cdot 17.7042) + 4.25 \cdot 0.0578^2 \\
&= 0.51 \cdot (313.438577 - 16.710658) - 147.39 \cdot 1.023503 + \\
&\phantom{= (} + 4.25 \cdot 0.003341 \\
&= 0.51 \cdot 296.727919 - 150.854707 + 0.014199 \\
&= 151.331239 - 150.854707 + 0.014199 \\
&= 0.490731.
\end{align*}

Тогда:
\begin{align*}
\Delta_4 &= 289.0425 \cdot 0.11422725 - 4.25 \cdot 0.490731 \\
&= 33.018932 - 2.085607 \\
&= 30.933325 > 0.
\end{align*}

\textbf{Определитель $\Delta_5$:}
\[
\Delta_5 = a_5 \cdot \Delta_4 = 4.25 \cdot 30.933325 = 131.466631 > 0.
\]

\textbf{Вывод:} Все определители Гурвица положительны:
\begin{gather*}
\Delta_1 = 0.51 > 0, \quad \Delta_2 = 0.51 > 0, \quad \Delta_3 = 0.11422725 > 0, \\
\Delta_4 = 30.933325 > 0, \quad \Delta_5 = 131.466631 > 0.
\end{gather*}

Следовательно, по критерию Гурвица замкнутая система \textbf{устойчива}.

\section*{Ответ:}

Система \textbf{устойчива}, так как все определители Гурвица положительны, следовательно, все корни характеристического полинома замкнутой системы имеют отрицательные вещественные части, что по критерию Гурвица означает устойчивость системы.

\vfill

\begin{flushright}
Семенов Илья \\
группа №3 \\
$i = 17$
\end{flushright}